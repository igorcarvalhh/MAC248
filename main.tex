\documentclass{article}

\usepackage[portuguese]{babel}
\usepackage{hyperref}

\title{Cálculo Diferencial e Integral IV (MAC248)\\Notas de Aula}
\author{Igor Carvalho Ramos Leal\\\href{mailto:igorcarvalho@poli.ufrj.br}{\texttt{igorcarvalho@poli.ufrj.br}}}
\date{Universidade Federal do Rio de Janeiro\\2023.1\\\today}

%\renewcommand{\thechapter}{\Roman{chapter}}
\newcommand{\aula}[2]{
	\newpage
	\section{#1 (#2)}
}

\begin{document}

\maketitle
\tableofcontents

%\aula{Modelos conceitual, logico e fisico.}{29/03/2023}

\section{Séries Infinitas}

\subsection{Definição de Séries Numéricas}
\subsection{Condição Necessária para Convergência de uma Série Infinita}
\subsection{Séries Infinitas de Termos Positivos: Teste da Comparação, Teste da Integral, Teste de D'Alembert (teste da razão)}
\subsection{Séries Alternadas: Teste de Leibniz (teste da série alternada)}
\subsection{Séries Absolutamente e Condicionalmente Convergentes}
\subsection{Séries de Potência: Definição, Intervalo de Convergência, Diferenciação e Integração de Séries de Potências.}
\subsection{Séries de Taylor}

\section{Solução por Séries de Equações Lineares de Segunda Ordem}
\subsection{Soluções por Sëries Próximo a Ponto Ordinário}
\subsection{Solução por Séries Poximo a Ponto Singular Regular (Método de Frobenius)}

\section{Transformadas de Laplace}

\subsection{Definição da Transformada de Laplace}
\subsection{Transformada de Laplace como transformação linear}
\subsection{Resolução de Problemas de Valor Inicial para Equações Diferenciais}
\subsection{Função Degrau}
\subsection{Propriedades da Transformada de Laplace}
\subsection{Resolução de Equações Diferenciais com Função Forçada Descontínua}
\subsection{'Função' Delta de Dirac e sua Transformada de Laplace}
\subsection{A Integral de Convolução}

\section{Problemas de Valores de Contorno}
\subsection{Problema de autovalores}
\subsection{Problema de Sturm Liouville (opcional)}

\section{Séries de Fourier}
\subsection{Definição}
\subsection{Teorema de Convergência de Fourier}
\subsection{Séries de Senos e Cosenos de Fourier}

\section{Equações Diferenciais Parciais(EDP) Clássicas}
\subsection{Classificação}
\subsection{Método de Separação de Variáveis}
\subsection{Equações do Calor:}
\subsubsection{Condições de Contorno: Dirichlet,Neumann, mista e Robin.}
\subsection{Equação da Onda:}
\subsubsection{Condições de Contorno: Dirichlet,Neumann e mista.}
\subsection{Equação de Laplace:}
\subsubsection{Condições de Contorno: Dirichlet e Neumann no Retângulo e Dirichlet no Círculo.}

\end{document}