%\aula{Modelos conceitual, logico e fisico.}{29/03/2023}

\aula{Sequências}{04/04/2023}

\subsubsection*{Teoremas importantes}

\begin{itemize}
    \item Dado uma função $f$ tal que $f(n) = a_n$, $\forall n \in N$. Então, se $\displaystyle \lim_{x \rightarrow \infty} f(x) = L \Longrightarrow \lim a_n = L$
    \item Se $\lim |a_n| = 0$, então $\lim a_n = 0$
    \item Dado a sequência $b_n = f(a_n)$. Se $\lim a_n = L$ e $f$ contínua em $L$, então $\lim b_n = \lim f(a_n) = f(\lim a_n)$
\end{itemize}

Exercícios

\begin{multicols}{2}
    \begin{enumerate}
        \item \[ \lim \frac{n}{\sqrt{10 + n}} \]
        \item \[ \lim \frac{n!}{n^n} \] Dica! pensar em $MA \geq MG$
        \item \[ \lim \frac{n^2}{e^n} \]
        \item \[ \lim \frac{n!}{2^n} \]
        \item \[ \lim \left( 1 + \frac{2}{n} \right)^2 \]
    \end{enumerate}
\end{multicols}